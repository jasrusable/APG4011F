\documentclass{article}

\usepackage{graphicx}
\usepackage{amsmath}

\title{APG4011F Assignment 3}
\date{07/05/2015}
\author{Jason David Russell - RSSJAS005}

\begin{document}

\maketitle
\pagenumbering{gobble}

\newpage
\tableofcontents

\pagenumbering{arabic}

\newpage
\section{Introduction}
The purpose of this assignment is to gain an understading into the principles of image restituion and bundle adjustment.
A python program will be used to demonstrate and simulate how image restituion and bundle adjusment is performed.

\section{Background}
Bundle adjustment can be defined as the simultaneous redining od 3D coordiantes describing a scene geometery as well as the parameters of the realtive motion and optical characteristcs of the cameras used to accuire images, accodng to an optially reiterion involving the correspsoding image projections of all points.


\section{Problem Statement}
There are three main questions which will be addressed in this assignment. They are listed below:

\subsection{Intersection}
Given a set of object points which have homolougous points in two seperate images, with each image having unique exterior orientation paramters, set up a least squares adjustment using the collieanarity equations to redetermine the object points from each pair of homologous points from each image. Thereafter, compare the new object coordiantes to those original, pregenrated object coordinates.

\subsection{Resection}
Given a set ob object points, which each have a homologus point in two seperate images, set up a least squares adjustment to determine the exterior orientation parameters of each image

\subsection{Bundle Adjustment}


\section{Method}

\section{Results}

\section{Discussion}

\section{Conclusion}

\end{document}
